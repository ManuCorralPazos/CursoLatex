\documentclass[11pt]{beamer}
\usepackage[utf8]{inputenc}
\usepackage[T1]{fontenc}
\usepackage{lmodern}
\usepackage[spanish]{babel}

\usepackage{amsmath}
\usepackage{amssymb}
\usepackage{amsfonts}
\usepackage{amsthm}
\usepackage{amscd}
\usepackage{latexsym}

\usetheme{Boadilla}


\begin{document}
	\author{Manuel Corral Pazos}
	\title{Mi primera presentación}
	%\subtitle{}
	\logo{\includegraphics[scale=0.05]{logoUSC.png}}
	%\institute{}
	\date{\today}
	%\subject{}
	%\setbeamercovered{transparent}
	%\setbeamertemplate{navigation symbols}{}
	\begin{frame}[plain]
		\maketitle
	\end{frame}
	
	\begin{frame}
		\frametitle{Guion}
		\tableofcontents
	\end{frame}
	
\section{Primera sección}
	\begin{frame}
		\frametitle{Primera}
		Hola qué tal estamos?
	\end{frame}

	\begin{frame}
		\frametitle{Segunda}
		Hoy hace un día estupendo
	\end{frame}
\section{Segunda sección}
	\begin{frame}
		\frametitle{Tercera}
		\begin{itemize}
			\item Probamos a meter un itemize
			\item A ver cómo queda
			\item En la presentación
		\end{itemize}
	Va solo
	\end{frame}

\subsection{Matemáticas}
	\begin{frame}{Ecuación}
		
		Ponemos una ecuación para referenciar
		\begin{equation} \label{eq:1} 
			\sum_{i=0}^{\infty} a_i x^i
		\end{equation}
		
		Ecuación \ref{eq:1} referenciada
		
	\end{frame}

	\begin{frame}{Diagrama}{Ejemplo CD}
		Hacemos un diagrama tipo Gauss con el cd
		\[
		\begin{CD}
			\left(\begin{array}{rr|r}
				1&3&1\\
				-2&-6&-2
			\end{array}\right) \pause @>>{\begin{matrix}
					f_2+2f_1\\1f_1
			\end{matrix}}> \pause
			\left(\begin{array}{rr|r}
				1&3&1\\
				0&0&0
			\end{array}\right) \pause \Rightarrow x_1=1-3x_2
		\end{CD}
		\]
	\end{frame}
\end{document}