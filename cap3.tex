\chapter*{Tercero}
\addcontentsline{toc}{chapter}{Tercero}

\section[Repetido del primero]{Primera sección, Martes mañana}
Esta es la primera sección.
\subsection[Colores]{Algunos colores}
Esta es una subsección de la primera sección\\
Probamos a escribir en color, por ejemplo \textcolor{blue}{\bf esto iría en azul}. \textcolor{red}{\bf Esto iría en rojo.} Le ponemos el bf para que aparezca en negrita~\index{negrita}. Ahora utilizamos el \textcolor{moradito}{\bf color que definimos en la cabecera.}


\begin{lema}
	Sea K un cuerpo. El cojunto $\mathbb{M}_{m\times n}(K)$, junto con la suma definida tiene estructura de grupo abueliano.
\end{lema}


\subsection[Escritura]{Algo de escritura}
Aquí ponemos algún array
\[
\left (
\begin{array}{rrr|r} 
	-1 & 2 &  3 &   0 \\ 
	3 & 4 & -7 &   2\\
	6 & 5 & 90 & -11
\end{array} 
\right )
\]

Como vimos en la tabla \ref{tb:tabla1}, que aparece en la página \pageref{tb:tabla1}


