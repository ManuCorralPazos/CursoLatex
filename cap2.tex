\chapter{Segundo}

\section[Martes tarde]{Tercera seccion, martes por la tarde}
En esta sección vamos a ver matrices, arrays, etc.
Por ejemplo, ponemos una matriz de la forma rrr|r

\[
\left (
\begin{array}{rrr|r} 
	-1 & 2 &  3 &   0 \\ 
	3 & 4 & -7 &   2\\
	6 & 5 & 90 & -11
\end{array} 
\right )
\]

\begin{minipage}{.8\textwidth} \color{blue}
	Por ejemplo, ponemos una matriz de la forma rrr|r
	
	\[
	\left (
	\begin{array}{rrr|r} 
		-1 & 2 &  3 &   0 \\ 
		3 & 4 & -7 &   2\\
		6 & 5 & 90 & -11
	\end{array} 
	\right )
	\]
\end{minipage}

\subsection{Sistema de ecuaciones}
Ponemos ahora un sistema de ecuaciones

\[
\left \{
\begin{array}{rr} 
	x+y-2z & = 5 \\
	-3x+5y-4z & = 0 \\
	3x+y+z & = 8
\end{array}
\right .
\]

\subsection{Otro sistema de ecuaciones}
De otra forma
\[
\left \{
\begin{array}{r}
	x+y-2z  = 5 \\
	-3x+5y-4z  = 0 \\
	3x+y+z = 8
\end{array}
\right .
\]

\begin{restatable}{theorem}{propMatr} \label{thm:propMatr}
	Sean $\forall A, B, C\in \mathcal{M}_{m\times n}(\mathcal{R})$ y $\forall \alpha, \beta\in \mathbb{R}$, se verifica:
	\begin{multicols}{2}
		\begin{enumerate} 
			\item $A+B=B+A$
			\item $\alpha(\beta A)=(\alpha \beta)A$
			\item $(A+B)+C=A+(B+C)$
			\item $(\alpha+\beta)A=\alpha A + \beta B$
		\end{enumerate}
	\end{multicols}
\end{restatable}

\section[Miercoles mañana]{Miércoles por la mañana}

Operador límite:

Calcula $\lim_{x\to \infty}\frac{1}{x^2+1}$ y la suma de la serie $\sum_{n=1}^{\infty}\frac{1}{n^2+1}$

\subsection{Centrando}

El mismo límite centrado:

\[
\lim_{x\to \infty}\frac{1}{x^2+1} \hspace{2cm} \sum_{n=1}^{\infty}\frac{1}{n^2+1}\hspace{2cm}
\]

En una línea con displaystyle $\displaystyle\lim_{x\to \infty}\frac{1}{x^2+1}$, y la serie $\displaystyle \sum_{n=1}^{\infty}\frac{1}{n^2+1}$

Utilizando la función LIM $\limito_{x\to \infty^{n\neq\dot{2}}\frac{1}{x^2+1}}$, con displaystyle $\displaystyle\limito_{x\to \infty^{n\neq\dot{2}}\frac{1}{x^2+1}}$

Ahora ponemos LiM como operatorname $\operatorname{LiM}_{x\to \infty^{n\neq\dot{2}}\frac{1}{x^2+1}}$, con displaystyle $\displaystyle\operatorname{LiM}_{x\to \infty^{n\neq\dot{2}}\frac{1}{x^2+1}}$

Calcula la unión de los conjuntos $\cup_{n=1}^{\infty} [n,n+1)$

Calcula la unión de los conjuntos $\cap_{n=1}^{\infty} [n,n+1)$

\subsection{Paquete CD}
Hacemos un diagrama con CD
\[
\begin{CD} 
	A @>j>> B @>>c> C \\
	@VVV @AAhA @| \\
	D @= E @>>p> F 
\end{CD}
\]

Para pegarlo a la esquina
$\displaystyle \begin{CD}
	A @>j>> B @>>c> C \\
	@VVV @AAhA @| \\
	D @= E @>>p> F 
\end{CD}$

Hacemos un diagrama tipo Gauss con el cd
\[
\begin{CD}
	\left(\begin{array}{rr|r}
		1&3&1\\
		-2&-6&-2
	\end{array}\right) @>>{\begin{matrix}
			f_2+2f_1\\1f_1
	\end{matrix}}>
	\left(\begin{array}{rr|r}
		1&3&1\\
		0&0&0
	\end{array}\right) \Rightarrow x_1=1-3x_2
\end{CD}
\]

\subsection{Alguna pmatrix}

Escribe que $\mathcal{B} = \{
\begin{pmatrix}
	1 & 0 \\ 1 & 0
\end{pmatrix},
\begin{pmatrix}
	1 & 2 \\ 0 & 0
\end{pmatrix},
\begin{pmatrix}
	1 & 0 \\ 2 & 1
\end{pmatrix},
\begin{pmatrix}
	0 & 1 \\ 1 & 0
\end{pmatrix}	
\}$ es una base del espacio vectorial $\mathcal{M}_2(\mathbb{R})$. Se pide, aunque no se pida nada;
Sabemos que $\exists A \in \mathcal{M}_2(\mathbb{R})$ tal que 
\[[A]_{\mathcal{C}}=\begin{pmatrix} 
	4\\0\\4\\1
\end{pmatrix}
\]
Y $\nexists M \in \mathcal{M}_2(\mathbb{R})$ tal que $[M]_{\mathcal{B}}=\begin{pmatrix}
	4\\0\\4\\1
\end{pmatrix}$
\vspace{2cm}

Calcula $1+2+3+\cdots + 25$, en vez de $2+4+6+\dots$
\vspace{2cm}

\subsection*{Entornos de enumeración}
\textbf{Itemize}
\begin{itemize}
	\item Resuelve $\lim_{x\to \infty}\frac{1}{x^2+1}$
	\begin{itemize}
		\item Calcula el límite anterior cuando $x\to 5$
	\end{itemize} 
	\item Resuelve $\lim_{x\to \infty} x$
\end{itemize}

\textbf{Itemize con etiqueta cambiada}
\begin{itemize}
	\item[1] Resuelve $\lim_{x\to \infty}\frac{1}{x^2+1}$
	\begin{itemize}
		\item Calcula el límite anterior cuando $x\to 5$
	\end{itemize} 
	\item Resuelve $\lim_{x\to \infty} x$
\end{itemize}

\textbf{Enumerate}
\begin{enumerate}
	\item Resuelve $\lim_{x\to \infty}\frac{1}{x^2+1}$
	\begin{enumerate} \setcounter{enumii}{4} %empiezo en e)
		\item Calcula el límite anterior cuando $x\to 5$
	\end{enumerate} 
	\item Resuelve $\lim_{x\to \infty} x$
\end{enumerate}

Continuamos con la lista

\begin{enumerate} \setcounter{enumi}{2} %empiezo en 3
	\item Resuelve $Ax+b=c$
	\item Resuelve $1+3x=y$
\end{enumerate}

\textbf{Enumerate con etiqueta cambiada}
\begin{enumerate}
	\item[$\bullet$]Resuelve $\lim_{x\to \infty}\frac{1}{x^2+1}$
	\begin{enumerate}
		\item Calcula el límite anterior cuando $x\to 5$
	\end{enumerate} 
	\item Resuelve $\lim_{x\to \infty} x$
\end{enumerate}

\textbf{Description}
\begin{description}
	\item[Casa] Lugar para vivir personas. Queremos que esto tenga más de una línea, para ver cómo queda al compilar. No llegó con eso, ponemos un poco más de texto.
	\item[Tienda] Lugar en el que se compra. También queremos que tenga más de una línea, a ver si esta vez somos capaces de conseguirlo.
	
\end{description}

Ahora usamos el paquete enumerate

\textbf{Enumerate}
\begin{enumerate}[i).-]
	\item Resuelve $\lim_{x\to \infty}\frac{1}{x^2+1}$
	\begin{enumerate}[1;]
		\item Calcula el límite anterior cuando $x\to 5$
		\item Calcula el límite anterior cuando $x\to 10$
	\end{enumerate} 
	\item Resuelve $\lim_{x\to \infty} x$
\end{enumerate}

\subsection{Creación de comandos}

Comandos creados en el preámbulo de la página con \verb*|newcommand|

\begin{description}
	\item[Raiz] \raiz
	\item[Ecu 1 arg] \ecu{3}
	\item[Ecu 2 args] \ecudos{2}{6}
	\item[Ecu op] \ecuop[\phi+1]
	\item[Ecu op default] \ecuop
	\item[Ecu tres] \ecutres{1}{2}
	\item[Suc] \suc
	\item[Suce] \suce[A]
	\item[Suce sin arg] \suce
	\item[Suce+suce] \suce[a] + \suce[b] = \suce
	\item[Sin] $\sin(\alpha)$
\end{description}

\subsection{Entorno theorem}

\begin{teorema}[Euler] \label{thm:euler}
	Sean $a,m\in \mathbb{Z}$ tales que $\gcd(a, m)=1$, entonces \[a^{\phi(m)}\equiv_m1\]
\end{teorema}

Como se puede ver en el Teorema de Euler \ref{thm:euler}, queda demostrado.

\begin{teorema}
	Sean $\forall A, B, C\in \mathcal{M}_{m\times n}(\mathcal{R})$ y $\forall \alpha, \beta\in \mathbb{R}$, se verifica:
	\begin{multicols}{2}
		\begin{enumerate} 
			\item $A+B=B+A$
			\item $\alpha(\beta A)=(\alpha \beta)A$
			\item $(A+B)+C=A+(B+C)$
			\item $(\alpha+\beta)A=\alpha A + \beta B$
		\end{enumerate}
	\end{multicols}
\end{teorema}

\begin{restatable}{lema}{carath}Caratheodory.\label{lema:carath}
	Sea $f(z)$ una función analítica en la bola cerrada $B(0,r)$ con centro 0 y radio r. Sea $f'(z)$ una función que definimos para rellenar texto.
\end{restatable}

\begin{prop}[Dos cuerpos]
	Sean...
\end{prop}

\begin{corol}
	Para toda matriz A, existen matrices invertibles, producto de matrices elementales, E, F, tales que E $\cdot$ A, es la forma escalonada de A, y 
	\[
	EAF=\begin{pmatrix}
		1&\\
		& \ddots& & &  0 \\
		& & 1 & \\
		& & & 0 & \\
		& 0 & & & \ddots\\
		& & & & &0\\
	\end{pmatrix}
	\]
\end{corol}

\begin{lema}[Segundo lema]
	Sea...
\end{lema}

\begin{definicion}[Anillo] \label{definicion:anillo}
	Un anillo está formado por tres ingredientes $(A,+,\cdot)$, donde $A$ es un conjunto no vacío, $+ : A\times A \rightarrow A$ es una aplicación y $\cdot: A\times A \rightarrow A$ es una aplicación (producto); que satisfacen los siguientes axiomas:
	\begin{enumerate}
		\item  $(A,+)$ es grupo abeliano.
		\item "+" es asociativa; es decir, para cualesquiera elementos $ a,b,c\in A$ se tiene $a+(b+c)=(a+b)+c$.
		\item Exixste el (único) neutro para $\cdot$ que solemos escribir $1\in A$ tal que $1\cdot a=a\cdot 1=a, \forall a\in A$;
	\end{enumerate}
\end{definicion}

\begin{definicion}[Matriz] \label{definicion:matriz}
	Sea K un cuerpo y $m, n\in \mathbb{N}$. Una matriz de orden $m\times n$ (en este orden) es un array bidimensional de elementos de K dispuestos en fimas de m, y columnas de n elementos; es decir:
	\[
	\begin{pmatrix}
		a_{11}&a_{12}& \dots & a_{1n}\\
		\vdots &\vdots \\
		a_{m1}&a_{m2}& \dots & a_{mn}
	\end{pmatrix}
	\]
\end{definicion}

Hacemos referencia a la definición de matriz \ref{definicion:matriz}
Hacemos referencia a la definición de anillo \ref{definicion:anillo}
También hicimos una definición de Grupo abeliano \ref{definicion:abeliano} en una sección anterior.
Ponemos aquí negrita~\index{negrita} para poder hacer referencia.

%se pone ~ para que no salte de línea entre la palabra y la referencia
Ahora rescatamos el lema~\ref{lema:carath}, que está en la página~\pageref{lema:carath} como sigue:
\carath*

También rescatamos el teorema~\ref{thm:propMatr}, que está en la página~\pageref{thm:propMatr} como sigue:
\propMatr*

\section[Tablas y figuras]{Miércoles tarde: tablas y figuras}

\subsection{Tablas}
\begin{table}[h!] \label{tb:tabla1}
	\begin{center}
		
		\begin{tabular}{|c|r|c|c|cc|}
			\hline
			a & \multicolumn{2}{r|}{bloque} & c & d & e \\
			\hline
			11 & 22 & 33 & 44 & 55 & 66 \\
			
			0 & 1 & 0 & 1 & cosa & $x^{}4$ \\
			\hline
			33 & 44 & 55 & 66 & 222 & 3 \\
			\hline
		\end{tabular}
	\end{center}
	\caption{Tabla primera}
\end{table}
\begin{table}[h!] \label{tb:tabla2}
	
	\begin{center}
		\begin{tabular}{c|ccc}
			
			$x$ & 1 & 2 & 3 \\
			\hline \\ [-0.2cm]
			$f(x)=x+1$ & 2 & 3 & 4 \\
			
		\end{tabular}
	\end{center}
	\caption{Tabla segunda}
\end{table}
\begin{table}[h!]\label{tb:tabla3}
	\begin{center}
		\renewcommand{\arraystretch}{2}
		\begin{tabular}{|c|r|c|c|c|c|}
			\hline
			a & \multicolumn{2}{r|}{bloque} & c & d & e \\
			\hline
			11 & 22 & 33 & 44 & 55 & 66 \\ 
			\hline
			0 & 1 & 0 & 1 & cosa & $x^{}4$ \\
			\hline
			33 & 44 & 55 & 66 & 222 & 3 \\
			\hline
		\end{tabular}
	\end{center}
	\caption{Tabla tercera}
\end{table}
\renewcommand{\arraystretch}{1}
\begin{table}[h!] \label{tb:tabla4}
	\begin{center}
		
		\begin{tabular}{|c|r|c|c|c|c|}
			\hline
			a & \multicolumn{2}{r|}{bloque} & c & d & e \\
			\hline
			11 & 22 & 33 & 44 & 55 & 66 \\ 
			\hline
			0 & 1 & 0 & 1 & cosa & $x^{}4$ \\
			\hline
			33 & 44 & 55 & 66 & 222 & 3 \\
			\hline
		\end{tabular}
	\end{center}
	\caption{Tabla cuarta}
\end{table}

\subsection{Figuras}            
Ahora incluimos una figura, debe de estar en la misma carpeta que este archivo .tex

\begin{figure}[h!]
	\centering
	\includegraphics[width=0.5\linewidth]{grafica1}
	\caption{Gráfico de barras}
\end{figure}

\carath*